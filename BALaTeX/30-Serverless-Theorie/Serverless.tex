\subsection{Serverless}

Nachdem die wichtigsten Servicemodelle vorgestellt wurden, ist es wichtig den Begriff Serverless zu präzisieren.


Es ist zweifelsohne nicht möglich Dienste in der Cloud zu verwenden, ohne irgendeine Art von Server zu beanspruchen.
Der Cloud Provider abstrahiert diese jedoch so weit, dass sich Nutzer keine Gedanken über die Infrastruktur, das Betriebssystem oder selbst die Laufzeitumgebung mehr machen muss.
Wie bereits im zuvor beschriebenen Bild verlagert sich die betriebliche Verantwortung immer mehr zum Cloud Provider und der Anwender kann sich ganz auf seine Applikation konzentrieren.

Serverless ist eine Kombination aus den Ansätzen Function as a Service ergänzend mit Backend as a Service, welche bestimmte Regeln befolgen muss.
Die Eventbasierten Funktionen stellen die Logik der Anwendung dar und können im Zusammenspiel mit
Backend Diensten wie Datenbanken oder Authentifizierungsdiensten eine vollwertige Applikation erzeugen.



\subsubsection{Richtlinien für Cloud Provider }

Ab wann gilt etwas als Serverless?

Das Whitepaper von Amazon \glqq Amazon Web Services – Serverless Architectures with AWS Lambda\grqq{} \cite[Seitenzahl 1]{AWSWhitepaper}
stellt hierzu folgende vier Aspekte auf die zutreffen sollten:

1) \glqq No server management – You don’t have to provision or maintain any servers.\grqq
  \\ Die Verwaltung der Server und Laufzeitumgebung übernimmt der Cloud Provider.

2) \glqq Flexible scaling – You can scale your application automatically or by adjusting its capacity through toggling the units of consumption (for example, throughput, memory) rather than units of individual servers.\grqq
\\ Der Cloud Provider muss eine flexible und automatiserte Skalierung ermöglichen. Dabei können eventuell bestimmte Parameter für die Kapazität gesetzt werden.

3) \glqq High availability – Serverless applications have built-in availability and fault tolerance.\grqq
\\ Der Cloud Provider stellt für seinen Dienst eine Hochverfügbarkeit und Fehlertoleranz. automatisch zur Verfügung.

4) \glqq No idle capacity – You don't have to pay for idle capacity.\grqq{} [...] \glqq There is no charge when your code isn’t running. \grqq
\\ Kosten dürfen nur bei Nutzung entstehen. Im Leerlauf kostet der Dienst nichts.


\subsubsection{Richtlinien für Entwickler }

Auch Entwickler sollten sich an bestimmte Prinzipien und Leitlinien halten um bestmöglich von der Architektur profitieren zu können.

Zum einen sollte jeder geschriebene Code isoliert und unabhängig voneinander ausführbar sein, um so auch die Zustandslosigkeit gewährleisten zu können.
Zudem muss der Entwickler darauf achten alle wichtigen Daten außerhalb des Funktionskontextes persistent zu speichern, da sie beim nächsten Ausführen verloren gehen.
Die einzelnen Events dürfen dabei nicht voneinander abhängig sein und sollten eigenständig durchlaufen können.
Es muss ein Push basiertes Konzept entwickelt werden um Events ordnungsgemäß auslösen zu können.
Das bedeutet, dass die Funktion nicht selbst entscheiden soll in welchem Umfang sie eine Anfrage abarbeiten soll und ohne Benutzereingaben auskommen muss.

Der Code muss schnell ausführbar sein und dem Single-Responsibility-Prinzip folgen.
Laut dem Single-Responsibility-Prinzip darf eine Klasse oder Funktion immer nur eine einzige Verantwortung haben.
Es darf nie mehr als einen Grund geben eine Funktion auszuführen.
Verwendet wird das Prinzip um übersichtlichen und leicht erweiterbaren Code zu schaffen.
Je nach Cloud-Provider gibt es bei Funktionen häufig eine maximale Ausführungszeit.
AWS erlaubt seinem Lambda-Dienst eine maximale Zeit von 15 Minuten.
Auch der Speicherverbrauch wird bei AWS auf 3 GB RAM limitiert.
Wird die Zeit oder der Speicherverbrauch überschritten, terminiert die Ausführung der Funktion sofort.\cite[]{LambdaPreise}



\subsubsection{Vor- und Nachteile}
Serverless Architektur ermöglicht Entwicklern eine schnellere und einfachere Bereitstellung von Diensten in der Cloud.
Viele Anforderungen werden bereits vom Cloud Provider übernommen und müssen nicht beachtet werden.
Dazu gehört das Verwalten und Administrieren von Virtuellen Maschinen, Hochverfügbarkeit oder Fehlertoleranz.
Zudem gibt es keine Investitionskosten(Capex), sondern nur operative Kosten(Opex).
Die operativen Kosten lassen sich jedoch jederzeit steuern und auf Wunsch auch gänzlich beseitigen, zum Beispiel, falls das Projekt eingestellt wird.
Im Vorhinein ist es nicht notwendig Kapazitäten zu berechnen oder große finanzielle Risiken einzugehen.

Dadurch, dass bereits viele technische Voraussetzungen vom Cloud Provider übernommen werden, ist die Einstiegshürde bis zum Entwickeln geringer.
Weiter gibt es weniger Abhängigkeiten zu anderen Abteilungen, da jeder einzelne die gesamte Anwendung selbst erstellen und verwalten kann.
Die Realisierung von Projekten mithilfe von Serverless Architektur unterstützt dabei die DevOps-Philosophie und sorgt für schnellere Reaktionen auf Änderungswünsche.
DevOps setzt sich aus den Begriffen Development und Operations zusammen und hat mehrere Ansätze und Bedeutungen.
Eine mögliche Variante steht für das Zu­sam­men­schmel­zen von Entwicklung, Betrieb und den operativen Aufgaben einer Anwendung.
Das Ziel ist eine effektivere Zusammenarbeit aller Teilbereiche und eine erhöhte Qualität der Anwendung.
Neben der Realisierung ist auch eine Anpassung an den Markt erleichtert.
Folglich reduziert sich die Dauer bis zur Veröffentlichung von Testumgebungen oder sogar des gesamten Projektes, auch als Time-To-Market bekannt.
Nachdem eine Version abgeschlossen wird, startet daraufhin direkt die nächste Iteration.
Der Entwicklungsprozess ist kreisförmig anstatt linear.
\cite[]{DevOps}

Dadurch dass das Backend nur für einzelne Anforderungen genutzt wird, beinhaltet das Frontend bei Serverless Applikationen in der Regel mehr Funktionen und Logik.
Zum Beispiel kann direkt mit Drittanbieter APIs kommuniziert werden ohne das Backend nutzen zu müssen.
Das führt zu einem schnelleren Feedback für den Anwender und somit zu einer verbesserten Benutzererfahrung.

Da die Serverless Architektur noch nicht so lange im Einsatz ist wie die bewährten Alternativen, gibt es zum Beispiel noch nicht viele Security Tools oder Frameworks die diese Architektur voll unterstützen.
Das auf Security spezialisierte Unternehmen Checkpoint hat erst im Jahr 2019, durch Zukauf des Firma Protego, ein Tool angeboten.
Protego selbst wurde 2017 gegründet und spezialisiert sich auf Function as a Service Dienste wie AWS Lambda oder Azure Functions. \cite[]{Checkpoint}

Neben Protego gibt es zwar noch weitere Security Tools die mit Serverless Diensten umgehen können, jedoch unterstützen sie entweder noch nicht alle Cloud Provider oder sind in den Funktionalitäten eingeschränkt. \cite[]{Security}

Weiterhin ist es oft mit großem Aufwand verbunden eine bereits bestehende Applikation zu einer Serverless Architekur zu migrieren.
Oft muss der gesamte Ansatz neu konzipiert und umgesetzt werden, da die meisten bestehenden Anwendungen weder eventbasiert noch zustandslos sind.

Ein zusätzlicher Nachteil ist die fehlende Kontrolle über das unterliegende System.
Zwar wird einem hierdurch viel Arbeit abgenommen, jedoch kann es immer wieder einen Anwendungsfall geben, indem mehr Kontrolle über die Hard- und Software benötigt wird.
Zum Beispiel ist man an bestimmte Versionen der Programmiersprachen gebunden, die der Cloud Provider vorgibt.
Zudem ist es schwierig Serverless Funktionen von einem Cloud-Provider zum anderen zu migrieren, da jeder Cloud Provider eine unterschiedliche Implementierung besitzt und stark von weiteren Diensten abhängig ist.
Zum Beispiel ist AWS Lambda mit vielen weiteren Diensten integriert und verbunden.
Diese Funktionalität lässt sich nicht unverändert übertragen.
Hier findet ebenfalls ein Vendor Lock-In statt.


Insgesamt können Serverless Anwendungen erhebliche Vorteile im Vergleich zu klassischen Anwendungen haben, jedoch ist es nicht immer möglich oder profitabel auf diese zu setzen.
Vor allem bei größeren, bereits bestehenden Anwendung lohnt es sich nicht immer die Architektur zu wechseln, bei neuen und webbasierten Anwendungen kann die Architektur ihre Vorteile voll und ganz ausspielen.

\subsection{Eignung für die Bachelorarbeit}
Wie bereits im Abschnitt \frqq \ref{Motivation} \nameref{Motivation} \flqq{} erwähnt wurde beschäftigt sich die Abteilung Datacenter and Clouds seit längerem mit Cloud Computing und hat auch schon Projekte mit unterschiedlichen Servicemodellen realisiert.
Function as a Service wurde ebenfalls verwendet, jedoch bisher nicht im Rahmen einer Webanwendung mit einem eigenen Frontend.

Es besteht der Wunsch nach einer Webanwendung, die bestimmte Daten der unterschiedlichen Cloud Provider sammelt und in einem übersichtlichen Frontend anzeigt.
Dafür soll kein dedizierter Server benötigt werden oder hohe Kosten entstehen.

Eine Serverless Architektur eignet sich für diese Anforderungen optimal.
Es muss keine Virtuelle Maschine, kein Betriebssystem und auch keine komplexen Konfigurationen erstellt werden.
Die aufgestellten Anforderungen lassen sich ideal in isolierte Funktionen aufteilen.
So könnte pro Cloud Provider eine oder mehrere Funktionen bereitstellt werden, welche komplett autark arbeiten.
Der Umfang einer Funktion wäre etwa das Sammeln einer Kostenübersicht bei AWS.

Die gesammelten Daten würden in einer Datenbank gespeichert werden und anschließend von einem Frontend dargestellt.
Die Datenbank sollte im besten Falle, wie der Rest der Applikation, vom Cloud Provider verwaltet werden und erforderliche Kapazitäten selbst anpassen können.
Da es keine Bestandsdaten gibt, kann die Anwendung auf die Serverless Architektur ohne Probleme optimiert werden.
Für das Frontend stehen viele moderne Frameworks zur Verfügung die ohne langjährige Erfahrung genutzt werden können.
Desweiteren ist es nicht notwendig die Authentifizierung komplett selbstständig zu schreiben, da bereits passende Dienste und Komponenten existieren.

Auch aus Kostensicht ist die Anwendung bestens geeignet für eine Architektur ohne Serververwaltung.
Es gibt keine konstante Auslastung und bei Nichtnutzung fallen auch keine Kosten an.


Durch die Architektur ist man selbst in der Lage jeden einzelnen Schritt selbst zu entwerfen und umzusetzen, ohne Experte mit Tiefenwissen in allen Gebieten zu sein.
Viele mühselige Aufgaben werden durch den Cloud Provider übernommen und man kann sich vollumfänglich auf die Anwendung selbst konzentrieren.

Mit welchen Diensten die Anwendung realisiert wird und wie genau die Komponenten konstruiert werden, beschreiben die nächsten Kapitel.



