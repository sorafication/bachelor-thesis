\subsection{Serverless}

Nach dem die wichtigsten Servicemodelle vorgestellt wurden, ist es wichtig den Begriff Serverless
zu präzisieren.


Es ist zweifelsohne nicht möglich Dienste in der Cloud zu verwenden ohne irgendeine Art von Server zu beanspruchen.
Der Cloud Provider abstrahiert diese jedoch soweit, dass sich Nutzer keine Gedanken über
die Infrastruktur, das Betriebssystem oder Speicherverwaltung mehr machen muss.
Wie bereits im zuvor beschriebenen Bild verlagert sich die betriebliche Verantwortung immer mehr zum
Cloud Provider und der Anwender kann sich ganz auf seine Applikation konzentrieren.

Serverless ist eine Kombination aus den Ansätzen Function as a Service ergänzend mit Backend as a Service,
welche bestimmte Regeln befolgen muss.
Die Eventbasierten Funktionen stellen die Logik der Anwendung dar und können im Zusammenspiel mit
Backend Diensten wie Datenbanken oder Authentifizierungsdiensten eine vollwertige Applikation erzeugen.



\subsubsection{Richtlinien für Cloud Provider }

Ab wann gilt etwas als Serverless

Das Whitepaper von Amazon \flqq Amazon Web Services – Serverless Architectures with AWS Lambda\frqq
stellt hierzu vier Aspekte auf die zutreffen sollten. Diese lauten:

1) Der Anwender muss keine Server hochfahren oder warten. Auch ist er nicht für die Laufzeitumgebung der Software
zuständig.

2) Die Applikation wird automatisch nach Bedarf flexibel skaliert. Die Skalierung selbst erfolgt nicht anhand von Servereinheiten
sondern durch festlegen von Parametern wie Speicherverbrauch.

3) Es wird direkt vom Dienst eine Hochverfügbarkeit und Fehlertoleranz zur Verfügung gestellt. Jede Funktion oder Applikation die
einen Serverless Dienst benutzt ist per Default auch Hochverfügbar.

4) Es dürfen keine Kosten für im Idle Zustand entstehen. Wird der Code nicht ausgeführt kostet er auch nichts.
\cite[]{AWSWhitepaper}


\subsubsection{Richtlinien für Entwickler }

Auch Entwickler sollten sich an bestimmte Prinzipien und Leitlinien halten um bestmöglich von der Architektur profitieren zu können.

Zum einen sollte jeder geschriebene Code isoliert und unabhängig voneinander ausführbar sein, um so auch
die Zustandslosigkeit gewährleisten zu können.
Zudem muss man als Entwickler darauf achten alle wichtigen Daten außerhalb des Funktionskontextes persistent zu speichern, da sie beim nächsten Ausführen verloren gehen.
Die einzelnen Events dürfen dabei nicht voneinander abhängig sein und sollten eingenständig durchlaufen können.
Es muss ein Push basiertes Konzept entwickelt werden um Events ordnungsgemäß auslösen zu können.

Der Code muss schnell ausführbar sein und dem Single-Responsibility-Prinzip
folgen.Laut dem Single-Responsibility-Prinzip darf eine Klasse oder Funktion immer nur eine einzige Verantwortung haben.
Es darf nie mehr als einen Grund geben eine Funktion auszuführen. Verwendet wird das Prinzip um übersichtlichen und leicht erweiterbaren Code zu schaffen.
Cloud-Provider erlauben Funktionen häufig nur eine maximale Ausführungszeit von 15 Minuten. Auch der Speicherverbrauch darf je nach Anbieter nicht mehr
als 1,5GB bzw. 3GB RAM benötigen.
Wird die Zeit oder der Speicherverbrauch überschritten, terminiert die Ausführung der Funktion sofort.



\subsubsection{Vor- und Nachteile}
Serverless Architekur ermöglicht Entwicklern eine schnellere und einfachere Bereitstellung von Diensten in der Cloud.
Viele Anforderungen werden bereits vom Cloud Provider übernommen und müssen nicht beachtet werden.
Dazu gehört das Verwalten und Administrieren von Virtuellen Maschinen, Hochverfügbarkeit oder Fehlertoleranz.
Zudem gibt es keine Investitionskosten(Capex) sondern nur operative Kosten(Opex). Die operativen Kosten lassen sich jedoch
jederzeit steuern und auf Wunsch auch gänzlich beseitigen, zum Beispiel falls das Projekt eingestellt wird.
Im Vorhinein ist es nicht notwending Kapazitäten zu berechnen oder große finanzielle Risiken einzugehen.

Dadurch, dass bereits viele technische Voraussetzungen vom Cloud Provider übernommen werden, ist die Einstiegshürde bis zum Entwickeln geringer.
Weiter gibt es weniger Abhängigkeiten zu anderen Abteilungen mehr, da jeder einzelne die gesamte Anwendung selbst erstellen und verwalten kann.
Die Realisierung von Projekten mithilfe von Serverless Architketur unterstützt dabei die DevOps-Philosophie
\footnote{DevOps setzt sich aus den Begriffen Development und Operations zusammen. Der Begriff steht für
das zu­sam­men­schmel­zen von Entwicklung, Betrieb und den operativen Aufgaben einer Anwendung. Das Ziel ist eine effektivere
Zusammenarbeit aller Teilbereiche und eine erhöhte Qualität der Anwendung.} und sorgt für mehr Agilität.
Neben der Realisierung ist auch eine Anpassung an den Markt erleichtert.
Folglich reduziert sich die Dauer bis zur Veröffentlichung von Testumgebungen oder sogar des gesamten Projektes, auch
als Time-To-Market bekannt.\cite[]{DevOps}


Da die Serverless Architektur noch nicht so lange im Einsatz ist wie die bewährten Alternativen,gibt es zum Beispiel noch nicht
viele Security Tools oder Frameworks die diese Architektur voll unsterstützen. Das auf Security spezialisierte Unternehmen Checkpoint
hat erst im Jahr 2019, durch Zukauf des Firma Protego, ein Tool angeboten. Protego selbst wurde 2017 gegründet und spezialisiert
sich auf Function as a Service Dienste wie AWS Lambda oder Azure Functions. \cite[]{Checkpoint}

Neben Protego gibt es zwar noch weitere Security Tools die mit Serverless Diensten umgehen können, jedoch unterstützen sie
entweder noch nicht alle Cloud Provider oder sind in den Funktionalitäten eingeschränkt. \cite[]{Security}

Weiterhin ist es mit großem Aufwand verbunden eine bereits bestehende Applikation zu einer Serverless Architekur zu migrieren.
Der gesamte Ansatz muss neu konzipiert und umgesetzt werden, da die meisten bestehenden Anwendungen
weder eventbasiert noch zustandslos sind.

Ein zusätzlicher Nachteil ist die fehlende Kontrolle über das unterliegende System. Zwar wird einem hierdurch viel Arbeit abgenommen, jedoch kann es immer wieder
einen Anwendungsfall geben indem mehr Kontrolle über die Hard- und Software benötigt wird.

Dadurch dass das Backend nur für einzelne Anforderungen genutzt wird, beinhaltet das Frontend bei Serverless Applikationen
in der Regel mehr Funktionen und Logik. Zum Beispiel kann direkt mit Drittanbieter APIs kommuniziert werden ohne das Backend nutzen zu müssen.
Das führt zu einem schnelleren Feedback für den Anwender und somit zu einer verbesserten Benutzererfahrung.

Insgesamt können Serverless Anwendungen erhebliche Vorteile im Vergleich zu klassichen Anwendungen haben, jedoch
ist es nicht immer möglich oder profitabel auf diese zu setzen. Vor allem bei größeren, bereits bestehenden Anwendung lohnt es
sich nicht immer die Architektur zu wechseln, bei neuen und webbasierten Anwendungen kann
die Architektur ihre Vorteile voll und ganz ausspielen.

\subsection{Eignung für die Bachelorarbeit}
Wie bereits im Abschnitt \frqq \ref{Motivation} \nameref{Motivation} \flqq{} erwähnt wurde beschäftigt sich die Abteilung Datacenter and Clouds
seit längerem mit Cloud Computing und hat auch schon Projekte mit unterschiedlichen Servicemodellen realisiert. Function as a Service wurde ebenfalls
verwendet, jedoch bisher nicht im Rahmen einer Webanwendung mit einem eigenen Frontend.

Es besteht der Wunsch nach einer Webanwendung, die bestimmte Daten der unterschiedlichen Cloud Provider sammelt und in einem übersichtlichen Frontend anzeigt.
Dafür soll kein dedizierter Server benötigt werden oder hohe Kosten entstehen.

Eine Serverless Architektur eignet sich für diese Anforderungen optimal. Es muss keine Virtuelle Maschine, kein Betriebssystem und auch keine
komplexen Konfigurationen erstellt werden. Die aufgestellten Anforderungen lassen sich ideal in isolierte Funktionen aufteilen.
So könnte man zum Beispiel pro Cloud Provider eine oder mehrere Funktionen bereitstellen die komplett autark arbeiten. Der Umfang einer Funktion
wäre etwa das Sammeln einer Kostenübersicht bei AWS.

Die gesammelten Daten können in einer relationalen Datenbank gespeichert werden und anschließend von einem Frontend dargestellt werden.
Die Datenbank sollte im Besten Falle, genau wie der Rest der Applikation, vom Cloud Provider verwaltet werden und erforderliche Kapazitäten
selbst anpassen können.
Da es keine Bestandsdaten gibt, kann die Anwendung auf die Serverless Architektur ohne Probleme optimiert werden.
Für das Frontend stehen viele moderne Frameworks zur Verfügung die ohne langjährige Erfahrung genutzt werden können. Desweiteren ist es
nicht notwendig die Authentifizierung komplett selbstständig zu schreiben, da bereits viele Dienste und Komponenten existieren.

Auch aus Kostensicht ist die Anwendung bestens geeignet für eine Architektur ohne Serververwaltung. Es gibt keine konstante Auslastung
und bei Nichtnutzung fallen auch keine Kosten an.


Durch die Architektur ist man selbst in der Lage jeden einzelnen Schritt selbst zu entwerfen und umzusetzen, ohne Experte mit Tiefenwissen
in allen Gebieten zu sein. Viele mühselige Aufgaben werden durch den Cloud Provider übernommen und man kann sich vollumfänglich auf die
Anwendung selbst konzentrieren.

Mit welchen Diensten die Anwendung realisiert wird und wie genau die Komponenten konstruiert werden, beschreiben die nächsten Kapitel.



