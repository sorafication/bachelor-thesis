
%-------------------------------------------------------------
\section*{Zusammenfassung}
%-------------------------------------------------------------
Notwendig?

In dieser Bachelorarbeit werden folgende Themen behandelt:

- Cloud Computing und Servicemodelle

- Function as a Service

- Serverless Architektur

- Serverless Dienste beim Cloud Provider Amazon Web Services und Designentscheidungen

- Implementierung einer Webanwendung mit AWS Amplify und React

- Ausblick auf die Zukunft



%-------------------------------------------------------------
\section*{Abstract}
%-------------------------------------------------------------

Die vorliegende Bachelorarbeit beschäftigt sich im Detail mit Serverless Architektur und Function as a Service.
Zur Verständlichkeit werden die verschiedenen Servicemodelle des Cloud Computings vorgestestellt und verglichen.
Die Bachelorarbeit beschränkt sich auf den Cloud Provider Amazon Web Services und der entsprechenden Dienste, die für den Einsatz von Serverless Anwendungen zur Verfügung stehen.
Dabei werden AWS-Dienste wie Lambda, Cognito, AppSync, DynamoDB und Amplify genauer untersucht und bewertet.
Auf Basis der untersuchten Dienste wird ein Prototyp bei AWS entwickelt und implementiert. Die Programmiersprache ist bei der Implementierung NodeJS bzw. Javascript und React.
Das Ziel ist eine moderne Web Applikation für die Mitarbeiter der Mediengruppe RTL, die komplett auf Serverless Architektur basiert
und die in der Bachelorarbeit erwähnten Vorteile vollständig ausnutzen kann.




\clearpage
