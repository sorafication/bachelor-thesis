\subsection{Amazon Web Services Allgemein}

Bevor die relevanten Komponenten beschrieben werden, folgt eine kurze Übersicht zum Cloud Provider selbst.
Amazon Web Services ist einer der größten Cloud Computing Anbieter der Welt. Das Unternehmen gehört zu 100 Prozent zu Amazon Inc. und bot 2006 erstmals seine Dienste an. CEO ist seit Beginn an Andrew R. Jassy.
Zu den größten Kunden gehören Netflix, CocaCola, Spotify, Dropbox und viele weitere. \cite[]{AWSAllgemein}

Mittlerweile stehen mehr als 175 AWS-Dienste in aktuell 24 unterschiedlichen Regionen zur Auswahl.
Jede Region besitzt in der Regel drei eigene Verfügbarkeitszonen (englisch: Availability Zones) die geografisch voneinader getrennt sind.
Die Regionen sind auf alle Kontinente verteilt.
Auch in Deutschland wird mit dem Standort Frankfurt eine eigene Region angeboten.
Mit Hinblick auf Datenschutz und der Datenschutz-Grundverordnung soll Amazon die Vorgaben erfüllen können.
So speichert selbst die Bundespolizei Bodycam Aufnahmen in der Amazon Cloud.\cite[]{AWSPolizei}
Es wird berichtet, {}\glqq dass derzeit noch keine staatliche Infrastruktur zur Verfügung stehe, die die Anforderungen erfülle.\grqq{}
\cite[Abschnitt 1]{AWSPolizei}

Amazon Web Services bietet für viele Anwendungsfälle einen oder mehrere passende Services an.
Jedes im vorherigen Kapitel besprochene Servicemodell hat mehrere entsprechende AWS-Dienste und es werden laufend neue angekündigt.

Auch für den Bereich Serverless gibt es mehrere Dienste.
Die nachfolgenden Abschnitte beschäftigen sich mit dem Design der Webanwendung.
Um eine bestmögliche Entscheidung treffen zu können, werden die wichtigsten Dienste untersucht und auf Ihre Eignung geprüft.
Falls notwendig wird zuvor die zugrundeliegende Technologie erläutert, sodass ein Verständnis für den jeweiligen Dienst geschaffen werden kann.
Anhand der Ergebnisse folgt eine Entscheidung, welche Dienste im Anschluss für die Implementierung genutzt werden.




